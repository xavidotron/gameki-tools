% This can be called for a particular macro with something like
% -jobname=cJamesBond by defining \thing to a macro based on jobname.
% If the second character of jobname is not uppercase, we
% define \thing to expand to every PC because that probably means
% jobname is not overridden.  Alternately, you can select the run with
% -jobname=1 or -jobname=1-cJamesBond.  \GAMERUN needs to be set
% before gametex.sty is loaded, so this processing has to be early.
%
% By default, do every PC.
\def\parsedmacros{\EVERY{PC}{\ME{}\par}}%
\def\parsejobname#1#2#3\endparse{%
  \ifx#2-\relax
    % 2-cJamesBond
    \def\GAMERUN{#1}%
    \def\parsedmacros{\csname#3\endcsname{}}%
  \else\ifx#2\empty\relax
    % 2
    \def\GAMERUN{#1}%
  \else\ifnum`#2=\uccode`#2\relax
    % cJamesBond
    \def\parsedmacros{\csname#1#2#3\endcsname{}}%
  \else
    % casting (i.e., not overridden)
    % Just stick to the default.
  \fi\fi\fi
}
% The \emptys are so we always have 3 parameters to \parsejobname; if excess
% they just end up in the \csname harmlessly.
\expandafter\parsejobname\jobname\empty\empty\endparse
